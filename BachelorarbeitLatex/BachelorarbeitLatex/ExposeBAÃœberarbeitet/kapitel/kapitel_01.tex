
\chapter{General}

\section{Motivation}
People are becoming more interested in matters concerning physical and mental health. This is most likely attributed to the COVID-19 pandemic that has been circulating in recent years. \cite{.bahn-bonn}
Along with positive outcomes, such as increased care for fellow citizens \cite{.bahn-bonn} and greater awareness of health issues, the consistent growth of interest in health issues also caused problems.
With an increasing number of anxious and concerned patients, medical practices and general practitioners have long since exceeded their capacity limits and have reached their breaking point. \cite{.sok} This is also noticed by the patients: Overcrowded waiting rooms combined with long waiting periods and nerve-racking telephone loops are becoming the norm for doctor visits.

\section{Conceptual Formulation}
% Rückblick
The conceptual formulation with which this bachelor thesis will deal can be traced back to the preceeding situation. The population is fearful, caused by the COVID-19 pandemic, and doctors are reaching their limits.
% Probleme mit der Auslastung von Arztpraxen und von verängstigten Patienten
The resulting problems are of great importance. General practitioners are being forced to order patient stops and issue access bans. \cite{.sok} This also means that patients in need of immediate medical attention may be turned away and medical care may be denied. In addition to the concerned patients, the number of seriously (COVID-19) ill people has steadily increased: there have been approximately 146,000 deaths in Germany since the start of the pandemic (as of August 19, 2022). \cite{.rki}
% Resultat 
% Folgende Problemstellung
This imposes the question on how to address patients' concerns while also relieving the burden on doctors.

\section{Objective}
% Lösungsansatz
Digitalization provides a solution to this problem. Online consultation hours and online appointment scheduling have recently helped relieve medical practices.
% Zielsetzung Allgemein
Smartphones, in particular, are becoming an increasingly important part of our daily lives. The goal of this bachelor thesis is to provide a method for efficiently minimizing the aforementioned problems through the use of mobile applications. Such an application can provide advice to a worried user and help alleviate their fears. It should be noted that the goal of this application is not to replace a doctor.
% Zielsetzung Applikation
The application development goal is to create an optimal database structure as well as an accurate and high-performance algorithm for symptom and disease assignment.

\section{Method}
Dart is the programming language that will be used to develop this application, along with the Flutter framework. The following four sub-goals will form the basis for the processing method in order to specify the objective of this bachelor thesis.

\begin{description}
	% Begin: Anlegen der Datenbank
	\item \textbf{Determining an optimal data structure and implementing it in Firebase:}
	A database in Firebase should be created at the start of the development process. This requires determining an optimal data structure and implementing it in the form of possible symptom and disease data. After that, the database can be built and integrated into the Flutter application.
	 % Abschließende Nutzerumfrage: Würden Sie einer solchen Applikation vertrauen schenken und welche Kriterien tragen zur Antwort bei
	\item \textbf{Survey of application trustworthiness:}
	A survey should reveal whether the people polled would put their trust in such an application. Furthermore, criteria that can have a positive impact on the question asked, such as a professional graphical user interface, are gathered. The information drawn from the survey can then be used to develop the graphical interface of the application accordingly.
	% Entwickeln der grafischen Oberfläche
	\item \textbf{Devolopment of the user interface:}
	The database can then be used to create the application's graphical user interface. The most important aspect of this graphical interface is the questionnaire that allows the user to select and rate their symptoms on a scale.
	
	% Implementierung des Kernpunktes: Algorithmen
	\item \textbf{Implementation and testing of the algorithms:}
	This sub-goal will be the primary focus of the bachelor thesis.  Three algorithms for symptom weighting and disease assignment are developed to evaluate disease detection. Their scalability and performance are compared and their correctness is determined. Finally, using an evaluation process, the best-performing algorithm is built into the application.
 
\end{description}
During processing, the methodology of qualitative utility analysis is used to determine the best algorithm and data structure.

\section{Research Question}
The context discussed raises two research questions. The first, more important one, is following:
% Forschungsfrage
	\begin{center}
	\textbf{Which algorithmic approach to symptom evaluation is appropriate to be used in a mobile application for identifying disease patterns?}\\
	\end{center}
While the algorithmic method is the core of the work, understanding how the user interface must be created in order to persuade users to utilize this application is equally crucial.
This brings us to our second research question: 
	\begin{center}
	\textbf{What are the most important requirements for a user interface in order for users to use it before going to the doctor?}
	\end{center}


\section{State of Research}
% Forschungsstand Flutter
The state of research on symptom checker applications explicitly developed in the Flutter framework is very limited. Flutter, as a development kit released in 2018 \cite{.flutter}, is still relatively new to the open source market. The current state of research on symptom checker applications will be described in general in the following section.
%Bezüglich Symptom-Checker-Applikationen welche explizit im Framework Flutter entwickelt worden sind ist der Forschungsstand sehr mager gefüllt. Als Entwicklungs-Kit, das erst im Jahr 2018 veröffentlicht wurde \cite{.flutter}, ist Flutter noch relativ neu auf dem Open-Source-Markt. Im Folgenden wird der Forschungsstand zu Symptom-Checker-Anwendungen im Allgemeinen beschrieben werden.\\
% Hypothese
If successfully developed, an assumption to be made is that a trustworthy application with a correct algorithm will positively (or possibly negatively) influence user concern and thus save its users a trip to the doctor.\\
%Eine anzunehmende Hypothese bei erfolgreicher Entwicklung ist, dass eine vertrauenswürdige Applikation mit einem korrektem Algorithmus die Besorgnis der Nutzer positiv (oder möglicherweise negativ) beeinflusst und ihren Nutzern somit einen Arztbesuch ersparen kann. \\
% Autor Prof. Dr. Andres Sönnichsen: negativ
Doctor Andreas Sönnichsen, President of the Evidence-Based Medicine Network, strongly opposed the use of such apps in his article "Fluch oder Segen? Symptom Checker und Diagnostik-Apps." which was released in 2019. He cited several studies that showed such applications had a low level of correct diagnosis. He emphasized at the end of his work that symptom checker applications should only be used as a differential diagnostic tool in the hands of a doctor, if he is stuck with unclear symptoms that could possibly be due to a rare disease that the doctor does not identify due to a lack of knowledge. \cite{.ebm}\\
%Dr. Andreas Sönnichsen, Präsident des Evidence-Based Medicine Network, hat sich im September 2019 in seiner Arbeit ''Fluch oder Segen? Symptom Checker und Diagnostik-Apps'' sehr deutlich gegen den Einsatz solcher Anwendungen ausgesprochen. Er deckte mehrere Studien auf, die zeigten, dass solche Anwendungen einen geringen Grad an korrekter Diagnose zeigten.  Er betonte als Abschluss seiner Arbeit, dass er Symptom-Checker-Anwendungen nur "... als differentialdiagnostisches Tool in ärztlicher Hand, wenn man bei unklaren Beschwerden nicht weiterkommt, hinter denen möglicherweise eine seltene Erkrankung stecken könnte, auf die der Arzt aufgrund von Wissensdefiziten nicht kommt." \cite{.ebm} genutzt werden sollten.\\
% Aktuelles
The goal of a current joint project at the University of Tübingen called "CHECK.APP" is to the use of a symptom checker application from multiple perspectives and derive recommendations for actions. This research has yielded no credible knowledge to this date.
%Ein aktuelles Verbund-Projekt der Universität Tübingen namens "CHECK.APP" verfolgt derzeit das Ziel, "den Umgang mit einer Symptom-Checker-App mehrperspektivisch zu analysieren und Handlungsempfehlungen abzuleiten." \cite{.uni} Zum jetzigen Stand ist aus dieser Forschung noch kein offizieller Wissensgewinn gezogen worden.









