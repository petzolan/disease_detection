
\section{Vorläufige Gliederung}
\begin{enumerate}
	\item Abstract
	\item Einleitung
		\begin{enumerate}
		\item Problemstellung
		\item Motivation und Ziele
		\end{enumerate}
	\item Grundlegendes
		\begin{enumerate}
		\item Wahl der Entwicklungssprache 
		\end{enumerate}
	\item Anlegen einer Datenbank
	\begin{enumerate}
		\item Wahl der Datenbank
		\begin{enumerate}
			\item SQL
			\item NoSQL - Firebase
		\end{enumerate}
		\item Erarbeiten einer Datenstruktur
		\item Erstellen von Datenzusammenhängen
		\item Einbinden der Daten in die Datenbank
			\begin{enumerate}
				\item Redundanz in Firebase
			\end{enumerate}
		\item Einbinden der Datenbank in das Flutter Projekt

	\end{enumerate}
	\item Entwicklung
		\begin{enumerate}
		\item Grafische Oberfläche
			\begin{enumerate}
			\item Konzeption der grafischen Oberfläche
			\item Entwicklung der grafischen Oberfläche
			\end{enumerate}

	\item State Management in der Applikation
	\item Grundlegende Logik
		\begin{enumerate}
		\item Konzeption der logischen Zusammenhänge
		\item Umsetzung der logischen Zusammehänge
		\end{enumerate}
	\end{enumerate}
	\item Entwicklung der Match-Algorithmen
	\begin{enumerate}
		\item Der Symptom-Graph
		\item Gewichtung der Symptome innerhalb der Applikation
		\item Entwicklung der Algorithmen
		\item Bewertung der Algorithmen
		\begin{enumerate}
			\item Performance-Vergleich
			\item Skalierbarkeit
		\end{enumerate}
	\end{enumerate}
	\item Gesamtüberblick der Applikation
	\begin{enumerate}
		\item Testen der Applikation
		\item Umfrage: Würden Befragte diese Applikation nutzen und ihr Vertrauen schenken
	\end{enumerate}
	\item Fazit und Ausblick
		\begin{enumerate}
		\item Symptomerkennungsapplikationen in der Zukunft
		\item Einsatz von Flutter zur Entwicklung von Applikationen
		\item Performance Resultate
		\item Gesamtfazit
	\end{enumerate}
	\item Abkürzungsverzeichnis
	\item Literaturverzeichnis
	\item Tabellenverzeichnis
	\item Abbildungsverzeichnis

\end{enumerate}
\section{Zeitplan}

Die Gesamte Bearbeitungsdauer der Bachelorarbeit beträgt fünf Monate, also ungefähr 21 Wochen.
\begin{itemize}
	\item Vorbereitung: Woche 1 und 2 
		\begin{itemize}
			\item Anlegen und Einbinden der Datenbank, Woche 1 - 2
		\end{itemize}
	\item Entwicklung: Woche 2 bis 8
		\begin{itemize}
			\item Grafische Oberfläche, Woche 2 - 3
			\item State Management, Woche 2 - 3
			\item Logik, Woche 3 - 4
			\item Algorithmen, Woche 4 - 8
		\end{itemize}
	\item Befragung: Woche 8 bis 10
		\begin{itemize}
			\item Umfrage laufen lassen, Woche 8 - 10
		\end{itemize}
	\item Schreibphase: Woche 8 bis 17
		\begin{itemize}
			\item Schreiben der Bachelorarbeit, Woche 8 - 17
		\end{itemize}
	\item Abschlussphase: Woche 17 - 21
		\begin{itemize}
			\item Korrekturlesen (lassen), Woche 17 - 20
			\item Eigenständiges Korrekturlesen, Woche 21
		\end{itemize}
\end{itemize}

% Literatur anzeigen
\section{Literatur}
\printbibliography[heading=none]








