
%% ABSTRACT %%

\selectlanguage{english}
{
	\raggedbottom
	\centering
	\vspace{0.9cm}
	\large
	\textbf{ABSTRACT}
		\vspace*{\fill}
		
		With an increasing number of anxious and worried patients, medical practices and general practitioners have long since exceeded their capacity limits and reached their breaking point.
		This thesis focuses on how a mobile application, which carries out a symptom analysis using algorithmic methods, has to be designed and implemented. The developed prototype should serve as a relief for doctors and a calming option for patients. In order to achieve the described goals, the methodology of stakeholder analysis was used. Also system requirements were determined, and use cases were then created. Based on these results, a domain model could be created, which serves as the basis for a database structure for Firestore. After considering which API to use, the database was filled. The first design for the graphical user interface was created in combination with implementation possibilities within the Flutter framewokr. Three different methods to calculate the possible diseases were compared and evaluated. The classes created and the structuring of the project, helped implement clean design principles, such as the SOLID principles. Surveys have found that the application's graphical user interface plays an important role. The mock-ups made in this thesis go in the right direction regarding intuitiveness and usabiliy, which was found using the SUS methodology. It could not be determined whether the chosen algorithm has a high correctness rate due to lack of correct data. A bad point, however, was that the time it took to perform the calculation needs to be lowered.
		\newline \\
		In general, a user-friendly application can be developed using the approaches discussed. However, improvements should be made regarding the performance of disease detection.
	
	\adjustbox{minipage=0.8\textwidth}{
		\vspace{1.5cm} 
		  
	\par}
	\vspace*{\fill}
	\pagebreak

	\selectlanguage{ngerman}
	\vspace{0.9cm}
	\large
	
	\textbf{\centering ACKNOWLEDGEMENTS}
	\vspace*{\fill}

	Mein hauptsächlicher Dank gilt meinem Betreuer Professor Doktor Sebastion Apel für seine hilfreichen Anregungen während des gesamten Betreuungszeitraumes. Die Freiheit, die er mir bei der Wahl des Themas gelassen hat, war nicht selbstverständlich.
	\newline \\
	Auch die Teilnehmerinnen und Teilnehmer meiner Befragung haben durch ihre Auskunftbereitschaft und interessanten Beiträge meine Bachelorarbeit wesentlich mitgeprägt.
	\newline \\
	Letztlich richte ich auch ein Dankeschön an meine Kommilitonin  Frau Jenny Hofbauer für das ausführliche Korrekturlesen und dem Mut machen während der gesamten Arbeit.
	
	\vspace{1cm}
	\noindent\hfil\rule{0.5\textwidth}{.4pt}\hfil
	\vspace{1cm}
	
	My main thanks go to my supervisor Professor Doctor Sebastion Apel for his helpful suggestions during the whole period of supervision. The freedom he gave me in choosing the topic was not self-evident.
	\newline \\
	The participants in my survey also played a major role in shaping my bachelor thesis through their willingness to provide information and interesting contributions.
	\newline \\
	Finally, I would also like to thank my fellow student Ms. Jenny Hofbauer for her detailed proofreading and encouragement throughout the thesis.
	\vspace*{\fill}
	\pagebreak

	\large
	\leavevmode%
	\vspace*{\fill}
	\begin{center}
		\textbf{DECLARATION}
	\end{center}
			I hereby declare that I have written this bachelor thesis independently, have not yet submitted it elsewhere for examination purposes, have not used any sources or aids other than those specified and have marked literal and analogous quotations as such.
	
	\vspace*{4em}\noindent
	\hfill%
	\begin{tabular}[t]{c}
		\rule{10em}{0.4pt}\\ Angelina Petzold
	\end{tabular}%
	\hfill%
	\begin{tabular}[t]{c}
		\rule{10em}{0.4pt}\\ Date, Location
	\end{tabular}%
	\hfill\strut
	\vspace*{\fill}

	\pagebreak

\par}