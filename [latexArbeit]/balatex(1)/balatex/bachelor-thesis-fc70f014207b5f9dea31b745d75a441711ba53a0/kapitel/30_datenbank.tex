
\chapter{The Database}
The following chapter is devoted to creating the Firestore database for the application. After the successful data conception based on API responses, the data sets are integrated into Firestore via a Python script. This procedure is also documented using a notebook (Jupyter Notebook) and is described there in more detail.

\section{Options for adding Datasets using APIs}
The aim of this work is the conception and implementation of the described system. In order to achieve this goal, the application must access a database filled with relevant data. The information from an existing API is used here since no medically trained project participants could contribute their specialized expertise to the database. The API selection also influences the database's final data structure. For this purpose, the two most fitting APIs are considered, and their suitability for the project is determined.

\subsection{NHS Health A to Z API}
The NHS, or National Health Service, is the publicly funded healthcare system of the United Kingdom. It was established in 1948 and provides a wide range of medical services to the population of the UK, including general practitioners, hospitals, and community health services. The NHS website provides many different APIs, free for use \cite{.nhs}. The NHS Health A to Z API is the first API that will be considered to generate data for the database.
\newline \\
The API offers medical information about various diseases, their symptoms, and available treatments. A user account that is provided with a subscription key must be created in order to receive data from this API. The next step is to make a call with the hypertext transfer protocol (HTTP) to \textbf{https://api.nhs.uk/conditions}. An example of a possible request in the programming language Python looks like this:
\begin{lstlisting}[language=Python, caption={Example Python Request for the Health A to Z API}]
	URL = "https://api.nhs.uk/conditions/acne"
	HEADER = {
		"subscription-key": "YOUR_SUBSCRIPTION_KEY"
	}
	response_disease = requests.request("GET", URL, headers=HEADER)
	response_data = response_disease.json()
\end{lstlisting}
If the request was successful and the response contains the data in the JavaScript object notation (JSON) format. Appendix C.1 shows an abbreviated sample response from the API. A positive aspect of this API is that all data is described in great detail and a large amount of knowledge can be obtained in a single query. However, it must also be mentioned that the extent of the server response just mentioned entails the difficulty of storing the data accordingly in a separate database. For example, symptoms are supplied for a disease, but only in string format as a complete sentence. This means that a symptom is described in different ways in several diseases. In order to automatically scrape this data, one would have to recognize all variations in the description of a symptom. This is almost impossible with the amount of data that the Health A to Z API brings. Another limitation is that a maximum of 6 requests can be made to the interface per minute, which proves to be a a big problem regarding the duration of data retention.

\subsection{ApiMedic Symptom Checker API}
The ApiMedic API is the second interface to be considered. ApiMedic is powered by priaid \cite{.apimedic}, a company that combines medicine, IT, and business administration. Thanks to their highly specialized team composition, they offer expertise in all areas mentioned. API calls can be done with two different accounts:
\begin{itemize}
	\item \textbf{Sandbox API Account:}
	It is possible to get unlimited data via the sandbox account. However, the data supplied is only dummy data.
	\item \textbf{Live Basic API Account:}
	The live account allows one to get the actual medical data of the API. However, there is a limitation concerning the possible calls: ApiMedic only allows 100 calls per month to be made without charge, and further requests cost money.
\end{itemize}
Although the data that can be received from this API is less detailed than that of the NHS API, using the data to create a data structure is more straightforward. It is possible to acquire diseases, bodily components, and symptoms as well as proposed symptoms and symptoms based on each body part. The following code example shows a request made with the live account to retrieve all diseases, the response of the API can be found in appendix C.2.
\begin{lstlisting}[language=Python, caption={Example Python Request for the ApiMedic API (all issues)}]
	URL_ISSUES = "https://healthservice.priaid.ch/issues?token=YOUR_TOKEN"
	response_issues = requests.request("GET", URL_ISSUES)
	data_issues = response_issues.json()
\end{lstlisting}
\noindent
It is now possible to execute an API request that returns detailed information about each disease using the provided IDs. 
\begin{lstlisting}[language=Python, caption={Example Python Request for the ApiMedic API (single issue)}]
	URL_ISSUE = "https://healthservice.priaid.ch/issues/105/info?token=YOUR_TOKEN"
	reponse_issue = requests.request("GET", URL_ISSUE)
	data_issue = reponse_issue.json()
\end{lstlisting}
\noindent 
Appendix C.3 shows a sample response from the ApiMedic API. The value of the "PossibleSymptoms"-key returns an enumeration of all the disease symptoms. These symptoms are listed using the values of the "Name"-key for the respective symptom when querying all symptoms. This makes it easier to scrape the data accordingly and store it in a database.

\subsection{API Solution}
The question that now arises is which of the two APIs to choose. Both APIs have advantages and disadvantages. While the NHS API provides very detailed results, it is challenging to use the data for the purposes intended in this work. NHS also provides data on the causes of various symptoms and conditions, which can be an essential factor in making a diagnosis in the form of disease detection. ApiMedic delivers the data in an optimal format but far less detailed than NHS. One available option is to use the symptom data provided by ApiMedic as scrape material for the symptom list in the NHS. However, after an attempt to do so, only a minimal amount of symptoms has been recognized. This is because the same symptom is named differently in both APIs. Generating more appropriate scrape material would require a full inspection of all NHS API data to ensure getting a decent amount of data. This is not possible within the scope of this work but should be considered for future system optimizations. In the context of the bachelor thesis, the ApiMedic API is preferred from the point of view of a clean database structure.

\section{Data Structure}
The data structure of the database is based on the decision that the ApiMedic API will be the leading supplier of the data. The basic data structure can already be guessed from the domain model, which is shown in Figure 4.4. Firestore supports the following datatypes: String, Number, Boolean, Map, Array, Null, Timestamp, Geopoint, and Reference. With a closer look at the API's JSON responses, the data structures can be formed.

\subsection{Examination of the JSON-Structure of ApiMedic}
ApiMedic does not provide any actual documentation. Instead, they allow interested users to test the HTTP requests directly via the ApiMedic website. To access the API, an HTTP request must be sent. The URL of the endpoint, the HTTP method, and, if necessary, other parameters and headers must be specified. All inquiries can be made via request to the following URL: \textbf{https://healthservice.priaid.ch/} in connection with the corresponding path to the desired file resource. The API responses can then be extracted from the HTTP message body as a JSON object.

\subsubsection{Body Part}
\begin{itemize}
	\item \textbf{Get all Body Regions: /body/locations?token=\{your\_token\}}
	\newline
	If the request is successful (status code 200), all body regions stored in the API form the output.
		\begin{table}[H]
			\centering
			\begin{tabular}{ | c| c| c | } 
				\hline
				Name of Field& Content & Datatype \\ 
				\hline
				ID & ID of the body region & Integer \\ 
				\hline
				Name & Name of the body region & String \\ 
				\hline
			\end{tabular}
				\caption{Body Regions Response Data}
		\end{table}

	With the help of the ID of the body regions, the individual body parts of a region can be determined.
	\item \textbf{Get all Body Parts: /body/locations/\{body\_region\_id\}?token=\{your\_token\}}
	\newline
	If the request is successful (status code 200), all symptoms of a body location stored in the API form the output.
	\begin{table}[H]
		\centering
		\begin{tabular}{ | c| c| c | } 
			\hline
			Name of Field& Content & Datatype \\ 
			\hline
			ID & ID of the body location & Integer \\ 
			\hline
			Name & Name of the body location & String \\ 
			\hline
		\end{tabular}
	 \caption{Body Locations Response Data}
	\end{table}	

	\item \textbf{Get all Body Symptoms (with gender IDs): /body/locations/\{body\_region\_id\}/\{gender\_id\}\\?token=\{your\_token\}}
	\newline
	If the request is successful, all body locations stored in the API form the output. The IDs of the genders are values ranging from 0 to 3. 
		\scriptsize
		\begin{table}[H]
			\centering
		\begin{tabular}{ | c| c| c | } 
			\hline
			Name of Field& Content & Datatype \\ 
			\hline
			ID & ID of the symptom & Integer \\ 
			\hline
			Name & Name of the symptom & String \\ 
			\hline
			HasRedFlag & Critical Flag & Boolean \\ 
			\hline
			HealthSymptomLocationIDs & IDs of  body locations affected by this symptom & Array[Integer] \\ 
			\hline
			ProfName & Professional name of the symtom & String \\
			\hline 			
			Synonyms & Synonyms of the symptom & Array[String] \\
			\hline
		\end{tabular}
			 \caption{Body Location Symptoms Response Data}
		\end{table}
		\normalsize
\end{itemize}
Instead of storing all body areas, all explicit body parts and their possible symptoms will be stored in Firestore. Based on the information obtained from successful HTTP requests, the Firestore structure can be defined as shown below.
\newline
\textbf{Collection: body\_parts}
\begin{table}[H]
	\centering
	\begin{tabular}{ | c| c| c | } 
		\hline
		Name of Document-field& Content & Datatype \\ 
		\hline
		name & Name of the body part & String \\ 
		\hline
		symptoms & List of all symptom ids of the body part  & Array[String] \\ 
		\hline
	\end{tabular}
			 \caption{Final Body Part Collection Structure for Database}
\end{table}
\noindent
The IDs of the documents are formed from the name of the respective object. Here, spaces are replaced by a "\_" and the name string is stripped of leading and trailing spaces using the Python function .strip(). The id generation method should never be used for real-world applications. However, it simplifies the legibility in the context of this bachelor thesis. Firestore saves all the data in the form of documents, which can be created using .doc(\{document\_id\}).set(\{...\}). When performing this operation, the document is automatically assigned the specified ID, making it unnecessary to store the id in the document itself.

\subsubsection{Ressource Symptom}
\begin{itemize}
	\item \textbf{Get all Symptoms :  /symptoms?token=\{your\_token\}}
	\newline		
	If the request is successful, all symptoms stored in the API form the output.
	\begin{table}[H]
		\centering
		\begin{tabular}{ | c| c| c | } 
			\hline
			Name of Field& Content & Datatype \\ 
			\hline
			ID & ID of the symptom & Integer \\ 
			\hline
			Name & Name of the symptom & String \\ 
			\hline
		\end{tabular}
	\caption{Symptoms Response Data}
	\end{table}
	
	\item \textbf{Get all proposed Symptoms:  /symptoms/proposed?symptoms=[\{symptom\_ids\}]\newline\&gender=\{gender\_name\}\&year\_of\_birth=\{birthyear\_YYYY\}?token=\{your\_token\}}
	\newline
	Unlike the query regarding all symptoms of a body part, the values of the genders are not to be given as an ID but in the form of full names, i. e. "male" and "female".
	\begin{table}[H]
		\centering
		\begin{tabular}{ | c| c| c | } 
			\hline
			Name of Field& Content & Datatype \\ 
			\hline
			ID & ID of the proposed symptom & Integer \\ 
			\hline
			Name & Name of the proposed symptom & String \\ 
			\hline
		\end{tabular}
	\caption{Proposed Symptoms Response Data}
	\end{table}
\end{itemize}
\noindent
Since all IDs of the symptoms for each body part are already stored in each body part document, there is no need to add the body part ids to the symptom anymore. Because of that, all that is needed to be stored is the name of the symptom and the ids of the proposed symptoms.
\newline
\textbf{Collection: symptoms}
\begin{table}[H]
	\centering
	\begin{tabular}{ | c| c| c | } 
		\hline
		Name of Document-field& Content & Datatype \\ 
		\hline
		name & Name of the body part & String \\ 
		\hline
		proposed\_symptoms & List of all proposed symptoms IDs of the symptom  & Array[String] \\ 
		\hline
	\end{tabular}
	 \caption{Final Symptom Collection Structure for Database}
\end{table}

\subsubsection{Ressource Disease}
\begin{itemize}
	\item \textbf{Get all issues (diseases):  /issues?token=\{your\_token\}}
	\newline
	If the request is successful (status code 200), all diseases stored in the API form the output.
	\begin{table}[H]
		\centering
		\begin{tabular}{ | c| c| c | } 
			\hline
			Name of Field& Content & Datatype \\ 
			\hline
			ID & ID of the body region & Integer \\ 
			\hline
			Name & Name of the body region & String \\ 
			\hline
		\end{tabular}
		\caption{All Diseases Response Data}
	\end{table}
	With the help of the ID of the body regions, the individual disease can be determined.
	\item \textbf{Get a single issue:  /issues/\{issue\_id\}?token=\{your\_token\}}
	\newline
	If the request is successful (status code 200), the requested issue forms the output.
	\begin{table}[H]
		\centering
		\begin{tabular}{ | c| c| c | } 
			\hline
			Name of Field& Content & Datatype \\ 
			\hline
			Description & Description of the disease & String \\ 
			\hline
			DescriptionShort & Short description of the disease & String \\ 
			\hline
			MedicalCondition & Description of the symptoms & String \\ 
			\hline
			Name & Name of the disease & String \\ 
			\hline
			PossibleSymptoms & All symptoms of the disease, comma separated string & String \\ 
			\hline
			ProfName & Professional name of the disease & String \\ 
			\hline
			Synonyms & Synonyms of the disease & String \\ 
			\hline
			Treatment & Treatment steps for the disease & String \\
			\hline 
		\end{tabular}
	 \caption{Disease Response Data}
	\end{table}
\end{itemize}
For the purpose of this work, only the name, description, symptoms, and treatment recommendation are extracted and stored.
\newline \\
\textbf{Collection: diseases}
\begin{table}[H]
	\centering
	\begin{tabular}{ | c| c| c | } 
		\hline
		Name of Document-field& Content & Datatype \\ 
		\hline
		name & Name of the body part & String \\ 
		\hline
		description & List of all symptom ids & Array[String] \\ 
		\hline
		treatment & Treatment recommendation of the disease & String \\ 
		\hline
	\end{tabular}
			 \caption{Final Disease Collection Structure for Database}
\end{table}
\subsubsection{Advice}
The advice generated by doctors is not part of the ApiMedic API data resources. They are created directly by users (doctors) of the application and require a title, description, associated symptoms, and diseases.
 \newline
\textbf{Collection: advices}
\begin{table}[H]
	\centering
	\begin{tabular}{ | c| c| c | } 
		\hline
		Name of Document-field& Content & Datatype \\ 
		\hline
		name & Title of the advice& String \\ 
		\hline
		description & Description of the advice & String \\ 
		\hline
		symptoms & List of associated symptom ids & Array[String]\\ 
		\hline
		diseases & List of associated disease ids & Array[String]\\ 
		\hline
	\end{tabular}
\caption{Final Advice Collection Structure for Database}
\end{table}
\subsubsection{Doctor}
Doctors have the opportunity to register to get access to the data entries. To do this, they will be asked to enter an email and a password. Firestore allows all of this to happen without involving a dedicated collection of users. Nevertheless, it makes sense to do this since user data could be expanded later. An example is that records added by a doctor could also be referenced in his account. This requires a collection that stores the user data of the respective doctor in separate documents. Firestore automatically generates the user ID as a hash value during registration.
 \newline
\textbf{Collection: doctors}
\begin{table}[H]
	\centering
	\begin{tabular}{ | c| c| c | } 
		\hline
		Name of Document-field& Content & Datatype \\  
		\hline
		email & E-Mail-Adress of the doctor & String \\
		\hline
	\end{tabular}
	\caption{Final Doctor Collection Structure for Database}
\end{table}
\subsubsection{User, User Specified Symptom, Diagnose, SymptomIntensity and NovelityFactor}
The user-defined symptoms are not stored in the database but are only generated temporarily and locally in the system during the diagnosis. The reason for not saving the user-specific symptoms is that a user does not have to log in to the application. Therefore, no user document is created in the database through which the symptoms can be traced back to the user. Only the individual attributes of the object are stored in the diagnosis in order to be able to understand the diagnosis when it is called up again. The classes of users and diagnoses, as well as the enumerations symptoms, intensity, and novelty factor, will not be stored in the database. The diagnoses are to be stored locally on the smartphone, ensuring that no user's personal data is stored in the database.

\section{Inserting the Data into the Database}
Adding the records to the database requires a few steps, which have been summarized with the help of Jupyter Notebook. The notebook can be downloaded by following the link of the QR code in Appendix E. After the successful data generation. The database can now be integrated into the application. For this, the scheme provided by the Flutter developers is followed.